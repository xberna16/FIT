\documentclass[a4paper]{article}

\usepackage[czech]{babel}
\usepackage[utf8]{inputenc}
\usepackage{amsmath}
\usepackage{graphicx}
\usepackage[colorinlistoftodos]{todonotes}
\usepackage[utf8]{inputenc}
\usepackage{graphicx}
\usepackage{xcolor}
\usepackage{fancyvrb}
\renewcommand{\contentsname}{Obsah}



\centerline{\sc \LARGE Vysoké učení technické v Brně}
\vspace{.5pc}
\centerline{\sc \large Fakulta informačních technologií}

\vspace{5pc}
\centerline{\includegraphics[scale=.4]{fit}}
\vspace{5pc}

\centerline{\large{\textit{\textbf{IMP-Mikroprocesorové a vestavěné systémy}}}}
\vspace{1pc}
\centerline{\large{\textbf{2018/2019}}}

\vspace{5pc}
\centerline{\sc \Large ARM-FITkit3: Aplikace modulu RNGA}
\vspace{1pc}
\centerline{\sc \large Hynek Bernard}
\centerline{xberna16}
\centerline{17.12.2018}
\vspace{10pc}
\centerline{vedoucí projektu}
\centerline{\sc J. Strnadel}
\clearpage

\tableofcontents
\clearpage



\begin{document}


\section{Úvod}
\subsection{Zadání}
Seznamte se s principem tvorby vestavných aplikací v jazyce C založených na mikrokontroléru Kinetis K60 (s jádrem ARM Cortex-M4) fy Freescale v prostředí Kinetis Design Studio (KDS) nebo MCUXpresso.

\noindent V jazyce C (tak i dále) vytvořte projekt demonstrující možnosti modulu Random Number Generator Accelerator (RNGA) dostupného na čipu Kinetis K60 z desky platformy FITkit 3.

\noindent UPŘESNĚNÍ: 1) Vytvořte aplikaci umožňující generování náhodných čísel pomocí RNGA.

2) Uložte dostatečně reprezentativní množinu posloupností čísel generovaných modulem RNGA.

3) Na základě získaných posloupností vhodně analyzujte a zdokumentujte vybrané vlastnosti generovaných čísel, např. determinismus, délku cyklu, jednotnost, korelaci.

4) Identifikujte slabiny generátoru, navrhněte a implementujte mechanismus s cílem zmírnit zvolenou slabinu.

5) Schopnost mechanismu zmírnit zvolenou slabinu experimentálně prokažte.

\subsection{FitKit3}
FitKit3 je samostatný hardware, který obsahuje výkonný mikrokontrolér určený pro výuku studentů VUT FIT. Byl mi zapůjčen fakultou. Software se na mikrokontroler tvoří v jazyce C

\subsection{RNGA}
RNGA modul je digitální integrovaný obvod generující 32-bitová náhodná čísla. Je umístěný v mikrokontroleru K60DN512M10, který je součástí FitKitu3. Konfigurace registrů zajišťuje statisticky dobrá data, to znamená, že data vypadjí náhodně. Neexistuje žádný kryptografický důkaz o bezpečnosti takové generace náhodných dat\cite{documentation}. RNGA disponuje Entropy registrem, to znamená, že pro náhodná data se může vygenerovat tzv entropy (náhodná veličina) která určí výsledek dalších dat. Nejlepší je tuto hodnotu brát najednou z co nejvíce proměnlivých a těžko reverzně dohledatelných zdrojů (pohyb myši, přesný čas atd..)

\clearpage

\section{Návod}
\subsection{Požadavky}
Mnou naprogramovaný fitkit komunikuje přes sériové rozhraní (COM port). Proto je zapotřebí programu který dokáže číst výstup (například PuTTy).

\noindent V případě potřeby programování fitkitu je nutné použít Kinetis Design Studio. Jiné utility nejsou potřeba

\subsection{Nastavení}
\noindent Parametry pro sériovou linku jsou : Rychlost- 115200, data bits- 8, stop bits-1, parity- none, flow control- none
\subsection{Spuštění}
\noindent Po připojení stačí stisknout tlačítko SW3 na fitkitu (DOWN) a začnou se ihned generovat data, která se posílají na sériovou linku oddělena mezerou. Rovněž se rozsvítí dioda D12, která značí že program běží a dioda D9, která značí rozsvícením a zhasnutím ukončení výpisu čísla, začne problikávat. Běh se dá kdykoliv pozastavit znovustisknutím tlačítka SW3. Pro přidání algoritmu plnění entropy registru RNGA je možné stistknout tlačítko SW5 (UP), přičemž se rozsvítí dioda D10. Doplňující algoritmus se dá kdykoliv vypnout znovustisknutím tlačítka SW5.

\section{Implementace}
\subsection{RNGA}
Pro zprovoznění rgna je pro něj zapotřebí nastavit hodiny, to dělám ve funkci PortsInit(). Poté již jen nastavím ve funkci RNGInit() kontrolnímu registru hodnotu 0xB, čímž vyčistím a maskuji přerušení, LSB značí spuštění RNGA modulu. Poté se počká na inicializaci oscilátorů (funkce delay) a pokud není program pozastaven, začne generovat čísla a vypisovat je. Pokud je entropy=1 tak je ve funkci printInt() plněn registr entropie následujícím způsobem: 

Představme si že indexujeme náhodně vygenerované číslo zprava (pro číslo 3587250926 je číslo 6 na nultém indexu a 3 na devátém indexu)

První vyplnění entropy registru posune právě vygenerované číslo o součet hodnot nultého a prvního indexu, v tomto případě o 8

Druhé vyplnění posune číslo o hodnotu prvního a nultého indexu, tedy o 26

Třetí vyplnění posune číslo o hodnotu druhého indexu, tedy o 9

Čtvrté vyplnění posune číslo o hodnotu třetího indexu, tedy o 0

\noindent Výše uvedený algoritmus je moje vylepšení stávajícího, výsledky se neliší výrazně, ale množina vygenerovaná s přidáním mého algoritmu prošla testy, kterými neprošla množina bez algoritmu. Dále by se dal algoritmus upravit například přidáním konkrétního timestampu s co nejvyšší přesností. Rychlost generování dat do output registru nemusíme řešit, protože než se vypíše číslo (převod na chary a odeslání UARTEM) tak se vždy stihne vyplnit novou pseudonáhodnou hodnotou.


\subsection{UART5}
Pro zprovoznění UART5 je také zapotřebí nastavit hodiny + inicializovat port pro vysílač (PORTE[8]), taktéž je další nastavení provedeno ve funkci PortsInit(). Další nastavení probíhá ve funkci UARTInit(). Nejprve je nutné vypnout transmitter i receiver v registru C2. Nastavuji v registru BDL rychlost 115 200Bd a 1stop bit hodnotou 0x1A. Nuluji adresy registrů C1,C3,MA1,MA2,BDH, jelikož nevyužíváme matchování adres a používáme 8 data bitů bez parity. Nakonec v registru C2 nastavíme hodnotu 0x8 která zapíná vysílač, který je od této chvíle funkční.




\subsection{Diody a tlačítka}
Diody a tlačítka jsou nastavovány ve funkci PortsInit(), kde diody jsou na portuB a tlačítka na portuE. Pro svůj kód jsem se inspiroval cvičeními z laboratoří a tlačítka jsou konfigurována identicky jako v laboratoři číslo 2 (s přenesením na fitkit, to znamená změna názvu portu) pull up rezistory. Rovněž se pro tlačítka musí zapnout přerušení, aby se jimi dal program ovládat za běhu.

\noindent Pro tlačítka bylo nutné vytvořit handler přerušení, ve kterém se mění stavy diod D10 a D12 podle hodnoty kterou tlačítka nastavují (SW5 a D10 úprava algoritmu, SW3 a D12 spouštění generátoru). Dioda D9 označuje vypsání čísla na výstup a není s žádným tlačítkem v handleru spjata. Funkci diody D9 je možné vidět při průběhu přerušení kdy dioda na chvíli přestává svítit protože neprobíhá žádný výpis.

\subsection{Program}
Program je zapsán tak, aby stisk tlačítek nenarušoval výpis. Tlačítka upravují hodnoty boolů které určují v podmínkách co se má dále dít. Pro korektní výpis je vygenerované číslo z RNGA ve funkci printInt() bráno jak integer bez znaménka (32 bitové nezáporné číslo) které se rozebere po jednotlivých cifrách na chary a vyšle se na UART. Program je v nekonečném cyklu, aby se zamezilo ukončení.

\section{Závěr}
\subsection{Analýza generovaných dat}
Vygeneroval jsem nejprve množinu dat A s počtem 1461651 vzorků, poté množinu dat B s počtem 1694342 vzorků. Množina A byla generována bez entropy veličiny a množina B měla po celou dobu zapnutý můj entropy algoritmus.

\noindent V matlabu vizuálním porovnáním histogramů veličin vidíme, že hodnoty v množině B jsou rovnomněrněji rozložené narozdíl od A. U obou množin je poslední sloupec malý kvuli omezení velikosti uint32.

\begin{figure}[#1]
  \centerline{\includegraphics[width=20cm, height=9cm]{histogramA.eps}}
  \caption[HistogramA]{Histogram množiny A}
\end{figure}
\begin{figure}[#2]
  \centerline{\includegraphics[width=20cm, height=9cm]{histogramB.eps}}
  \caption[HistogramB]{Histogram množiny B}
\end{figure}
\clearpage
\noindent Další analýza proběhla opensourcovým programem pro analýzu generátorů náhodných čísel dieharder. Použil jsem verzi 3.31.1 subsystémem Ubuntu ve Windows 10. Vstupní soubory veličin potřebovaly úpravu (převedení mezery na nový řádek), jinak byla délka čísel v pořádku.

\noindent Analýza množiny A trvala 5hodin a 10 minut, Analýza množiny B trvala 7 hodin a 5 minut

\begin{Verbatim}[fontsize=\tiny]

        test_name   |ntup| tsamples |psamples|  p-value A |Assessment A    |  p-value B |Assessment B
#===========================================================================================================#===#
   diehard_birthdays|   0|       100|     100|0.92147554|  PASSED          |0.27450955|  PASSED  
      diehard_operm5|   0|   1000000|     100|0.00000000|  FAILED          |0.00017248|   WEAK   
  diehard_rank_32x32|   0|     40000|     100|0.21656629|  PASSED          |0.02911547|  PASSED  
    diehard_rank_6x8|   0|    100000|     100|0.09889171|  PASSED          |0.00013391|   WEAK   
   diehard_bitstream|   0|   2097152|     100|0.99757749|   WEAK           |0.08469590|  PASSED  
        diehard_opso|   0|   2097152|     100|0.00000000|  FAILED          |0.00000000|  FAILED  
        diehard_oqso|   0|   2097152|     100|0.10129135|  PASSED          |0.00094371|   WEAK   
         diehard_dna|   0|   2097152|     100|0.01924592|  PASSED          |0.10336967|  PASSED  
diehard_count_1s_str|   0|    256000|     100|0.63992805|  PASSED          |0.19670739|  PASSED  
diehard_count_1s_byt|   0|    256000|     100|0.84709485|  PASSED          |0.99826503|   WEAK   
 diehard_parking_lot|   0|     12000|     100|0.14615348|  PASSED          |0.48712953|  PASSED  
    diehard_2dsphere|   2|      8000|     100|0.25168448|  PASSED          |0.41281836|  PASSED  
    diehard_3dsphere|   3|      4000|     100|0.64013216|  PASSED          |0.94177610|  PASSED  
     diehard_squeeze|   0|    100000|     100|0.00000000|  FAILED          |0.00000000|  FAILED  
        diehard_sums|   0|       100|     100|0.00504053|  PASSED          |0.58579000|  PASSED  
        diehard_runs|   0|    100000|     100|0.26412050|  PASSED          |0.17151654|  PASSED  
        diehard_runs|   0|    100000|     100|0.09677491|  PASSED          |0.32800650|  PASSED  
       diehard_craps|   0|    200000|     100|0.00000000|  FAILED          |0.00000000|  FAILED  
       diehard_craps|   0|    200000|     100|0.00000000|  FAILED          |0.00000000|  FAILED  
 marsaglia_tsang_gcd|   0|  10000000|     100|0.00000000|  FAILED          |0.00000000|  FAILED  
 marsaglia_tsang_gcd|   0|  10000000|     100|0.00000000|  FAILED          |0.00000000|  FAILED  
         sts_monobit|   1|    100000|     100|0.33150034|  PASSED          |0.02724982|  PASSED  
            sts_runs|   2|    100000|     100|0.98571535|  PASSED          |0.29048340|  PASSED  
          sts_serial|   1|    100000|     100|0.19561699|  PASSED          |0.01953259|  PASSED  
          sts_serial|   2|    100000|     100|0.02613244|  PASSED          |0.84787634|  PASSED  
          sts_serial|   3|    100000|     100|0.41490236|  PASSED          |0.10399995|  PASSED  
          sts_serial|   3|    100000|     100|0.29987448|  PASSED          |0.01230726|  PASSED  
          sts_serial|   4|    100000|     100|0.43532482|  PASSED          |0.08174597|  PASSED  
          sts_serial|   4|    100000|     100|0.03841719|  PASSED          |0.18078597|  PASSED  
          sts_serial|   5|    100000|     100|0.00011255|   WEAK           |0.00007211|   WEAK   
          sts_serial|   5|    100000|     100|0.01406491|  PASSED          |0.00000027|  FAILED  
          sts_serial|   6|    100000|     100|0.00032240|   WEAK           |0.00002066|   WEAK   
          sts_serial|   6|    100000|     100|0.75595446|  PASSED          |0.00105339|   WEAK   
          sts_serial|   7|    100000|     100|0.16154872|  PASSED          |0.00001990|   WEAK   
          sts_serial|   7|    100000|     100|0.30869981|  PASSED          |0.92215355|  PASSED  
          sts_serial|   8|    100000|     100|0.09759497|  PASSED          |0.00634774|  PASSED  
          sts_serial|   8|    100000|     100|0.55248260|  PASSED          |0.33621405|  PASSED  
          sts_serial|   9|    100000|     100|0.54238391|  PASSED          |0.29038149|  PASSED  
          sts_serial|   9|    100000|     100|0.21024461|  PASSED          |0.05503963|  PASSED  
          sts_serial|  10|    100000|     100|0.03305507|  PASSED          |0.56041695|  PASSED  
          sts_serial|  10|    100000|     100|0.07993301|  PASSED          |0.00864060|  PASSED  
          sts_serial|  11|    100000|     100|0.36001851|  PASSED          |0.89808222|  PASSED  
          sts_serial|  11|    100000|     100|0.04329310|  PASSED          |0.29633514|  PASSED  
          sts_serial|  12|    100000|     100|0.02553918|  PASSED          |0.61101293|  PASSED  
          sts_serial|  12|    100000|     100|0.13311270|  PASSED          |0.05035934|  PASSED  
          sts_serial|  13|    100000|     100|0.42244600|  PASSED          |0.07884566|  PASSED  
          sts_serial|  13|    100000|     100|0.00298120|   WEAK           |0.00791281|  PASSED  
          sts_serial|  14|    100000|     100|0.11382237|  PASSED          |0.16307393|  PASSED  
          sts_serial|  14|    100000|     100|0.41330076|  PASSED          |0.33082106|  PASSED  
          sts_serial|  15|    100000|     100|0.30801310|  PASSED          |0.00111609|   WEAK   
          sts_serial|  15|    100000|     100|0.69127286|  PASSED          |0.71820994|  PASSED  
          sts_serial|  16|    100000|     100|0.52486871|  PASSED          |0.06792482|  PASSED  
          sts_serial|  16|    100000|     100|0.39235492|  PASSED          |0.66694056|  PASSED  
         rgb_bitdist|   1|    100000|     100|0.78129016|  PASSED          |0.02453174|  PASSED  
         rgb_bitdist|   2|    100000|     100|0.84185282|  PASSED          |0.46436623|  PASSED  
         rgb_bitdist|   3|    100000|     100|0.23864458|  PASSED          |0.07376870|  PASSED  
         rgb_bitdist|   4|    100000|     100|0.21442101|  PASSED          |0.67602972|  PASSED  
         rgb_bitdist|   5|    100000|     100|0.69013220|  PASSED          |0.78345056|  PASSED  
         rgb_bitdist|   6|    100000|     100|0.71135580|  PASSED          |0.97136258|  PASSED  
         rgb_bitdist|   7|    100000|     100|0.65800481|  PASSED          |0.68393026|  PASSED  
         rgb_bitdist|   8|    100000|     100|0.47688452|  PASSED          |0.68515976|  PASSED  
         rgb_bitdist|   9|    100000|     100|0.69302444|  PASSED          |0.40459667|  PASSED  
         rgb_bitdist|  10|    100000|     100|0.08665845|  PASSED          |0.28906927|  PASSED  
         rgb_bitdist|  11|    100000|     100|0.55870264|  PASSED          |0.97718494|  PASSED  
         rgb_bitdist|  12|    100000|     100|0.01044058|  PASSED          |0.46873469|  PASSED  
rgb_minimum_distance|   2|     10000|    1000|0.02079940|  PASSED          |0.00087730|   WEAK   
rgb_minimum_distance|   3|     10000|    1000|0.00067734|   WEAK           |0.25829049|  PASSED  
rgb_minimum_distance|   4|     10000|    1000|0.00020525|   WEAK           |0.00069560|   WEAK   
rgb_minimum_distance|   5|     10000|    1000|0.00060793|   WEAK           |0.00014993|   WEAK   
    rgb_permutations|   2|    100000|     100|0.73524528|  PASSED          |0.65664643|  PASSED  
    rgb_permutations|   3|    100000|     100|0.05075967|  PASSED          |0.88608619|  PASSED  
    rgb_permutations|   4|    100000|     100|0.20688088|  PASSED          |0.04667639|  PASSED  
    rgb_permutations|   5|    100000|     100|0.79143402|  PASSED          |0.00002300|   WEAK   
      rgb_lagged_sum|   0|   1000000|     100|0.00000003|  FAILED          |0.00000000|  FAILED  
      rgb_lagged_sum|   1|   1000000|     100|0.00000000|  FAILED          |0.00000000|  FAILED  
      rgb_lagged_sum|   2|   1000000|     100|0.00000000|  FAILED          |0.00000000|  FAILED  
      rgb_lagged_sum|   3|   1000000|     100|0.00000000|  FAILED          |0.00000000|  FAILED  
      rgb_lagged_sum|   4|   1000000|     100|0.00000029|  FAILED          |0.00002865|   WEAK   
      rgb_lagged_sum|   5|   1000000|     100|0.00000000|  FAILED          |0.00000000|  FAILED  
      rgb_lagged_sum|   6|   1000000|     100|0.00000003|  FAILED          |0.04753670|  PASSED  
      rgb_lagged_sum|   7|   1000000|     100|0.00000006|  FAILED          |0.00000000|  FAILED  
      rgb_lagged_sum|   8|   1000000|     100|0.00000000|  FAILED          |0.00000262|   WEAK   
      rgb_lagged_sum|   9|   1000000|     100|0.00000001|  FAILED          |0.00000000|  FAILED  
      rgb_lagged_sum|  10|   1000000|     100|0.00002495|   WEAK           |0.00000475|   WEAK   
      rgb_lagged_sum|  11|   1000000|     100|0.00000000|  FAILED          |0.00000000|  FAILED  
      rgb_lagged_sum|  12|   1000000|     100|0.00000000|  FAILED          |0.00000000|  FAILED  
      rgb_lagged_sum|  13|   1000000|     100|0.00000000|  FAILED          |0.00000000|  FAILED  
      rgb_lagged_sum|  14|   1000000|     100|0.00000000|  FAILED          |0.10472120|  PASSED  
      rgb_lagged_sum|  15|   1000000|     100|0.00000000|  FAILED          |0.00000000|  FAILED  
      rgb_lagged_sum|  16|   1000000|     100|0.00000000|  FAILED          |0.00000143|   WEAK   
      rgb_lagged_sum|  17|   1000000|     100|0.00000000|  FAILED          |0.00000000|  FAILED  
      rgb_lagged_sum|  18|   1000000|     100|0.00000000|  FAILED          |0.00054439|   WEAK   
      rgb_lagged_sum|  19|   1000000|     100|0.00000000|  FAILED          |0.00000000|  FAILED  
      rgb_lagged_sum|  20|   1000000|     100|0.00000000|  FAILED          |0.05982638|  PASSED  
      rgb_lagged_sum|  21|   1000000|     100|0.00000003|  FAILED          |0.00000000|  FAILED  
      rgb_lagged_sum|  22|   1000000|     100|0.00000000|  FAILED          |0.00000000|  FAILED  
      rgb_lagged_sum|  23|   1000000|     100|0.00000000|  FAILED          |0.00000000|  FAILED  
      rgb_lagged_sum|  24|   1000000|     100|0.00149692|   WEAK           |0.00031115|   WEAK   
      rgb_lagged_sum|  25|   1000000|     100|0.00000019|  FAILED          |0.00000000|  FAILED  
      rgb_lagged_sum|  26|   1000000|     100|0.00000000|  FAILED          |0.00000000|  FAILED  
      rgb_lagged_sum|  27|   1000000|     100|0.00000000|  FAILED          |0.00000000|  FAILED  
      rgb_lagged_sum|  28|   1000000|     100|0.00000000|  FAILED          |0.00025996|   WEAK   
      rgb_lagged_sum|  29|   1000000|     100|0.00000000|  FAILED          |0.00000000|  FAILED 
      rgb_lagged_sum|  30|   1000000|     100|0.00000004|  FAILED          |0.00237523|   WEAK  
      rgb_lagged_sum|  31|   1000000|     100|0.00007123|   WEAK   
      rgb_lagged_sum|  32|   1000000|     100|0.00000000|  FAILED  
     rgb_kstest_test|   0|     10000|    1000|0.57890008|  PASSED  
     dab_bytedistrib|   0|  51200000|       1|0.00000000|  FAILED  
             dab_dct| 256|     50000|       1|0.02178273|  PASSED  
Preparing to run test 207.  ntuple = 0
        dab_filltree|  32|  15000000|       1|0.00000000|  FAILED  
        dab_filltree|  32|  15000000|       1|0.00000000|  FAILED  
Preparing to run test 208.  ntuple = 0
       dab_filltree2|   0|   5000000|       1|0.00000000|  FAILED  
       dab_filltree2|   1|   5000000|       1|0.00000000|  FAILED  
Preparing to run test 209.  ntuple = 0
        dab_monobit2|  12|  65000000|       1|1.00000000|  FAILED  
\end{Verbatim}

\vspace{1pc}
\noindent Ve výstupu programu je zřejmé že množina B uspěla ve více testech než množina A. Konkrétně si vedla lépe v 8 případech. To mě utvrzuje v názoru že můj algoritmus pomohl "náhodnosti" čipu RNGA.
\noindent Generátor bude při každém spuštění generovat jiná data, proto je možné že se objeví situace kdy i perfektní opravdu náhodný generátor neprojde některými testy, ve většině případů je zvládnutelná tolerance 5\%\cite{dieharder}
\subsection{Vyjádření}
Generátor náhodných čísel na FitKitu3 mi přijde rychlý a s dobrou základní funkčností. Osobně si myslím že hodnoty jsou v celku rovnoměrně rozložené a je pro vývojáře příjemné nemuset implementovat generátor náhodných čísel softwarově, ale s pomocí oscilátorů. Pokud by byla data potřebná na složité simulace či výherní automaty, rozhodně bych nedoporučil data z tohoto čipu, jinak pro použití ve hrách a nenáročných aplikacích jsou data dostačující.

\begin{thebibliography}{9}

\bibitem{documentation}
K60 Sub-Family Reference Manual 

Document Number: K60P144M100SF2V2RM

Freescale Semiconductor, Inc
\bibitem{dieharder}
dieharder (1) - Linux Man Pages


https://www.systutorials.com/docs/linux/man/1-dieharder/

\end{thebibliography}
\end{document}